%\documentclass[a4,semhelv,landscape]{seminar}
\documentclass[landscape]{slides}
%\documentclass[pdf, default, slideBW, nocolorBG]{prosper}
\usepackage[left=0.2cm,top=0.2cm,right=0.2cm,bottom=0.2cm,nohead,nofoot]{geometry}
%\def\everyslide{\sffamily}
%\usepackage{fullpage}
\usepackage{graphicx}
\usepackage[usenames]{color}
%\usepackage{color}
\usepackage{verbatim}
\usepackage{nopageno}
\usepackage{setspace}
%\usepackage{times}
% define some nice colors
\definecolor{myred}{rgb}{0.6,0,0}
\definecolor{myblue}{rgb}{0,0.2,0.4}
\definecolor{mygreen}{rgb}{0,0.5,0.0}
\definecolor{mypurple}{cmyk}{0.5,1.0,0.0,0.0}
\definecolor{myorange}{cmyk}{0.0,0.75,1.0,0.0}
%\color{myblue}

\begin{document}
%%%%%%%%%%%%%%%%%%%%%%%%%%%%%%%%%%%%%%%%%%%%%%%%%%%%%%%%%%%%%%%%%%%%%%
\begin{slide}
\begin{center}
\textbf{Rfam: RNA families database (3016 families)}

\small
\begin{itemize}
\item Each family is represented by:
  \begin{itemize}
  \item representative SEED alignment annotated with secondary-structure
  \item covariance model (CM) built from the SEED
  \item hits in Rfamseq database above GA threshold (FULL)
  \end{itemize}
\end{itemize}

\vspace{0.5in}
\includegraphics[height=5in]{figs/kalvari18-rfam-schema}

\vfill
\tiny \flushleft{Kalvari et al., 2018}
\end{center}
\end{slide}
%%%%%%%%%%%%%%%%%%%%%%%%%%%%%%%%%%%%%%%%%%%%%%%%%%%%%%%%%%%%%%%%%%%%
\begin{slide}
\begin{center}
\includegraphics[height=7in]{figs/kalvari-17-fig1}
\end{center}
\vfill
\tiny \flushleft{Kalvari et al., 2018}
\end{slide}
%%%%%%%%%%%%%%%%%%%%%%%%%%%%%%%%%%%%%%%%%%%%%%%%%%%%%%%%%%%%%%%%%%%%
\begin{slide}
\begin{center}
\includegraphics[height=7in]{figs/kalvari-2017-fig7}
\end{center}
\vfill
\tiny \flushleft{Kalvari et al., 2018}
\end{slide}
%%%%%%%%%%%%%%%%%%%%%%%%%%%%%%%%%%%%%%%%%%%%%%%%%%%%%%%%%%%%%%%%%%%%
%%%%%%%%%%%%%%%%%%%%%%%%%%%%%%%%%%%%%%%%%%%%%%%%%%%%%%%%%%%%%%%%%%%%%%
\begin{slide}
\begin{center}
  \textbf{Rfamseq: switch from subset of ENA to genome-centric database}
  \begin{itemize}  
  \item about 8000 reference genomes
  \item reduces redundancy
  \item more scalable
  \end{itemize}
\vspace{0.5in}
\includegraphics[height=4in]{figs/kalvari18-rfam-schema}
\end{center}    
\vfill
\tiny \flushleft{Kalvari et al., 2018}
\end{slide}
%%%%%%%%%%%%%%%%%%%%%%%%%%%%%%%%%%%%%%%%%%%%%%%%%%%%%%%%%%%%%%%%%%%%%%
\begin{slide}
\begin{center}
%  \textbf{Rfamseq: switch from subset of ENA to genome-centric database}
%  \begin{itemize}  
%  \item about 8000 reference genomes
%  \item reduces redundancy
%  \item more scalable
%  \end{itemize}
  \textbf{Rfamseq: genome-centric database means less flexible SEEDs}
  \begin{itemize}  
  \item previous requirement: SEED sequences must be in Rfamseq
  \item new: any GenBank or RNAcentral sequence can be in the SEED
  \item verification of sequences utilizes GenBank and RNAcentral API
  \end{itemize}
\vspace{0.5in}
\includegraphics[height=4in]{figs/kalvari18-rfam-schema}
\end{center}    
\vfill
\tiny \flushleft{Kalvari et al., 2018}
\end{slide}
%%%%%%%%%%%%%%%%%%%%%%%%%%%%%%%%%%%%%%%%%%%%%%%%%%%%%%%%%%%%%%%%%%%%%%
\begin{slide}
\begin{center}
  \textbf{Rfam in the cloud}
  \small
  \begin{itemize}  
  \item Ioanna has made the Rfam family building pipeline available to anyone
  \item Students in Daniel Gautheret's class are building Staphylococcus RNA families
  \item We are seeking RNA experts to improve or add to Rfam
    \begin {itemize}
      \item new, relaxed SEED requirements make it easier to use existing curated alignments 
    \end{itemize}
  \end{itemize}
\end{center}    
\vfill
\end{slide}
%%%%%%%%%%%%%%%%%%%%%%%%%%%%%%%%%%%%%%%%%%%%%%%%%%%%%%%%%%%%%%%%%%%%%%
\begin{slide}
\begin{center}
  \textbf{Rfam in the cloud}
  \small
  \begin{itemize}  
  \item Ioanna has made the Rfam family building pipeline available to anyone
  \item Students in Daniel Gautheret's class are building Staphylococcus RNA families
  \item We are seeking RNA experts to improve or add to Rfam
    \begin {itemize}
      \item new, relaxed SEED requirements make it easier to use existing curated alignments 
    \end{itemize}
  \end{itemize}

  \normalsize
  \textbf{Future directions for Rfam}
  \small
  \begin{itemize}  
  \item improve Rfam families based on crystal structures
  \item synchronize with mirBase
  \item viral families from Manja Marz
  \item use model reference coordinates to annotate important features 
  \end{itemize}
\end{center}    
\vfill
\end{slide}
%%%%%%%%%%%%%%%%%%%%%%%%%%%%%%%%%%%%%%%%%%%%%%%%%%%%%%%%%%%%%%%%%%%%%%
\end{document}
%%%%%%%%%%%%%%%%%%%%%%%%%%%%%%%%%%%%%%%%%%%%%%%%%%%%%%%%%%%%%%%%%%%%
\begin{slide}

\begin{center}
\large{\textbf{Rfam is 16 years old}}
\end{center}

\bigskip

\scriptsize

%ref: ftp://ftp.sanger.ac.uk/pub/databases/Rfam/CURRENT/README
%ref: http://infernal.janelia.org
\begin{center}
\begin{tabular}{ccrrcc} \hline
     &         &    \# of & Infernal& Rfam & Infernal \\
year & release & families & version & head & team \\ \hline
2002 & 0.1     &        4 &     0.3 & SGJ  & SE  \\
2002 & 0.2     &       12 &     0.3 & SGJ  & SE  \\
2002 & 0.3     &       21 &     0.3 & SGJ  & SE  \\
2002 & 1.0     &       25 &     0.4 & SGJ  & SE  \\
2002 & 2.0     &       30 &     0.55 & SGJ  & SE  \\
2003 & 3.0     &       36 &     0.55 & SGJ  & SE  \\
2003 & 4.0     &      114 &     0.55 & SGJ  & SE  \\
2003 & 4.1     &      165 &     0.55 & SGJ  & SE  \\
2003 & 5.0     &      176 &     0.55 & SGJ  & SE  \\
2004 & 6.0     &      350 &     0.55 & SGJ  & SE  \\
2004 & 6.1     &      379 &     0.55& SGJ  & SE  \\
2005 & 7.0     &      503 &     0.55& SGJ  & SE  \\
%2004 & \multicolumn{5}{c}{My (EPN's) doctoral work on \sft{infernal} begins} \\
2007 & 8.0     &      574 &     0.7 & PG  & EN, DK, SE  \\
2007 & 8.1     &      607 &     0.81& PG  & EN, DK, SE  \\
2008 & 9.0     &      603 &     0.81& PG  & EN, DK, SE  \\
2008 & 9.1     &     1372 &     0.81& PG  & EN, DK, SE  \\
2010 & 10.0    &     1446 &     1.0  & PG  & EN, SE \\
2011 & 10.1    &     1973 &    1.0.2 & PG  & EN, SE \\
2012 & 11.0    &     2208 &   1.0.2 & SB, EN  & EN, SE \\
2014 & 12.0    &     2450 &   1.1.1 & SB, EN  & EN, SE \\
2016 & 12.1    &     2473 &   1.1.1 & AP  & EN, SE \\
2017 & 12.2    &     2588 &   1.1.2 & AP  & EN, SE \\
2017 & 12.3    &     2687 &   1.1.2 & AP  & EN, SE \\
2017 & 13.0    &     2686 &   1.1.2 & AP  & EN, SE \\
2018 & 14.0    &     2791 &   1.1.2 & AP  & EN, SE \\
2019 & 14.1    &     3016 &   1.1.2 & AP  & EN, SE \\
\end{tabular}

\end{center}

\tiny
SGJ: Sam Griffiths-Jones; SE: Sean Eddy; PG: Paul Gardner; EN: Eric
Nawrocki; DK: Diana Kolbe; \\ SB: Sarah Burge; AP: Anton Petrov
\normalsize


\vfill
\end{slide}
\begin{slide}
\begin{center}

\textbf{RNAcentral uses Rfam to annotate sequences}

\begin{itemize}
\item 16.5 million sequences
\item 50\% are from Rfam
\item 40\% have Rfam hits
\item  2\% are potential sources of new Rfam families
\end{itemize}

\end{center}
\vfill
\end{slide}

\begin{slide}

\large
\begin{center}
\large{\textbf{Acknowledgements}} \\

\normalsize
\vspace{0.75in}

\small
\begin{tabular}{l|l|l}
%                  & \\ \hline
%                  & \\
Ioanna Kalvari        & Sarah Burge      \\
Anton Petrov          & Paul Gardner     \\
Rob Finn              & Sam Griffiths-Jones \\
Alex Bateman          & Jen Daub            \\
Sean Eddy             & Diana Kolbe 
\end{tabular}

\includegraphics[width=2.5in]{figs/NIH_NLM_ABRV_2C_4-white}
\includegraphics[width=2.5in]{figs/ncbi-logo}

\end{center}

\vfill
\end{slide}

